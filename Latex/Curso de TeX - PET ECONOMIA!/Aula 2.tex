\documentclass[a4paper, 12pt]{article} %Tipo de documento.
\usepackage[top=2cm, bottom=2cm, left=2.5cm, right=2.5cm]{geometry} %Pacote para Margens
\usepackage[utf8]{inputenc} % Pacote para permitir a utilização de acentos.
\usepackage{amsmath, amsfonts, amssymb, dsfont} % % Pacote matemático
\usepackage{graphicx} % Pacote inserir imagens.
\usepackage[portuguese]{babel} % Pacote para usar diversos idiomas
\usepackage{float} %Para inserir textos nas imagens
\usepackage{tikz} % Pacote para gráficos Tikz
\usepackage{tabu} %Pacote para tabular
\usetikzlibrary{arrows} % Pacote para o Tikz
\usepackage{setspace} %Pacote para espaçamento entre linhas
\usepackage{cases} % Para definir ordens ao modelo matemático
\usepackage[normalem]{ulem} % Pacote para usar o uuline (texto sublinhado)
\usepackage{multirow} % Pacote para criação de Tabular.
\title{Aula 02} %Inserir Título
\author{Felipe Ferreira e Victor Gabriel} %Inserir autor



\begin{document} %Inicio do ambiente do documento.
\maketitle %Inserir título pré-definido no preâmbulo.


\begin{flushleft} %Começar um texto pela esquerda.
\textbf{Introdução ao modo Matemático} %Texto em negrito
\end{flushleft} %Finalizar o ambiente que iniciou no begin.


 	\paragraph{}O Tex tem uma forma padrão de se incluir componentes matemáticos. Num parágrafo, o componente matemático é incluido pela seguinte maneira. $\backslash$( e $\backslash$) ou $\backslash$begin$\lbrace$math$\rbrace$ e $\backslash$end$\lbrace$math$\rbrace$. % Parágrafo. 
 
 		\subparagraph{} Sendo $a$ e $b$ os catetos e $c$ a hipotenusa de um triângulo retângulo, então: $ c^{2} = a^{2} + b^{2} $ %Componente matemático quando houver $$, por exemplo $a$. Para inserir a potência é fazer "c^2".
 
			\paragraph{} As fórmulas matemáticas maiores ou as equações t em uma melhor apresentação em linhas separadas do texto, para isso escreve-se o texto matemático entre $\backslash$ e [$\backslash$] ou entre $\backslash$begin$\lbrace$displaymath$\rbrace$ e $\backslash$end$\lbrace$displaymath$\rbrace$. Isto produz fórmulas sem número de equação. %Utilização do parágrafo.

				\subparagraph{} Sendo $a$ e $b$ os catetos, e $c$ a hipotenusa de um triângulo retângulo, então \begin{displaymath} c = \sqrt{a^{2} + b^{2}}\end{displaymath} %Utilização do sub-parágrafo.

				\paragraph{} Com $\backslash$label e $\backslash$ref pode-se fazer referência a uma equação dentro do corpo do texto. % Parágrafo.
\\
\begin{equation}\label{eq:eps} %Iniciando uma equação. O comando label é pra se referir ao arquivo.
\epsilon > 0                   %comando de seta para informar que é maior do que 0.
\end{equation} % Final do ambiente de equação.
De (\ref{eq:eps}) se deduz\ldots  %Código referente a colocar refêrencias na sua equação e o \ldots é sobre inserir três pontos. 

\paragraph{} Existem diferenças entre o modo matemático e o modo texto. Por exemplo, no modo matemático. % Parágrafo. 

\begin{flushleft} %Começar um texto pela esquerda.
\textbf{Agrupando no modo matemático:} %Texto em negrito
\end{flushleft} %Finalizar o ambiente que iniciou no begin.

\paragraph{} No modo matemático a maioria das instruções só afeta o carater seguinte. Se desejar que uma instrução influa sobre vários caracteres, então deve agrupá-los usando chaves. ($\lbrace$...$\rbrace$) % Parágrafo.

\subparagraph{} %Sub-parágrafo.
\begin{equation} %Inicio de equação.
a^x+y \neq a^{x+y} % Equação. Simbolo \neq é referente a indiferença.
\end{equation} % Final da equação.

\begin{flushleft} 
\textbf{Elementos das fórmulas matemáticas:}
\end{flushleft}

\paragraph{} Nesta seção são descritas as instruções mais importantes que se utilizam nas fórmulas matemáticas. % Sub-parágrafo.

\subparagraph{As letras gregas minúsculas:}se introduzem como $\backslash$alpha, $\backslash$beta, $\backslash$gamma... e as maiúsculas se introduzen como $\backslash$Gamma, $\backslash$Delta... %Exemplos de utilização de alguns comandos.

\subparagraph{}
\begin{equation}
\lambda, \xi, \pi, \phi, \omega
\end{equation}

\begin{equation}
\Lambda, \Xi, \Pi, \Phi, \Omega
\end{equation}

\subparagraph{Os expoentes e os subíndices.} 
\begin{equation}
a {1} \qquad x^{2} \qquad
e^{-\alpha t} \qquad a^{3} {ij}\\
e^{x^2} \neq {e^x}^2
\end{equation}

\subparagraph{O sinal de raiz quadrada }se introduz com $\backslash$sqrt, e a raiz n-ésima com $\backslash$sqrt[n].

\begin{equation}
\sqrt{x} \qquad
\sqrt{x^{2}+\sqrt{y}} \qquad
\sqrt[3]{2}
\end{equation}

\subparagraph{} Existem funções matemáticas (seno, coseno, tangente, logarítimo...) que se apresentam com letra arredondada. Para essas funções Tex proporciona as seguintes instruções:

\subparagraph{}

\[ \lim {n \rightarrow 0}
\frac{\sin x}{x}=1\]

\subparagraph{} Uma fração se faz com o comando $\backslash$frac$\lbrace$numerador$\rbrace$ $\lbrace$denominador$\rbrace$. Para as funções simples às vezes é preferível utilizar o comando $\backslash$. 
\\
\begin{center}
$\frac{1}{2}$~horas
\end{center}
\begin{displaymath}
\frac{x^{2}}{k+1} \qquad
x^{\frac{2}{k+1}} \qquad x^{1/2}
\end{displaymath}

\subparagraph{} Os coeficientes dos binômios e estruturas similares se podem criar com os comandos $\lbrace$... $\backslash$choose ...$\rbrace$ ou $\lbrace$...$\backslash$atop...$\rbrace$. Com o segundo comando conseguese o mesmo, apenas sem os parênteses.

\begin{displaymath} % Utilização do display matemático para realização de formulas.
{n \choose k}\qquad {x \atop y+2}
\end{displaymath}

\subparagraph{} O sinal de integral se obtém com $\backslash$int e o sinal de somatório com $\backslash$sum.

\begin{displaymath}  % Utilização do display matemático para realização de formulas.
\sum {i=1}^{n} \qquad  % Código da Somatória.
\int {0}^{\frac{\pi}{2}} \qquad % Código da Integral.
\end{displaymath}

\begin{flushleft} 
\textbf{Matrizes} % Texto em negrito.
\end{flushleft}

\paragraph{} Para compôr matrizes e similares existe no TEX o ambiente array. Este funciona de modo similar ao ambiente tabular. Usa-se o comando para mudar de coluna e para dividir as linas se utiliza a instrução $\backslash$ $\backslash$.

\begin{displaymath}
\mathbf{X}= % Deixando o nosso caracter matemático.  em negrito.
\left( \begin{array}{cccc}   % \left deixando o nosso caracter na esquerda. \begin{array} iniciando a matriz. {cccc} Definir colunas.
x_{11} & x_{12} & x_{13} & x_{14}\\ % Inserinndo os números da matriz. O Símbolo & é para inserir um espaço entre as linhas.
x {21} & x {22} & x{23} & x {24}\\ % o código \\ é para pularmos de linha.
x {31} & x {32} & x {33} & x {34}\\
x {41} & x {42} & x {43} & x {44}\\
\end{array}\right)
\end{displaymath}

\subparagraph{} Também se pode usar o ambiente array para compor expressões de funções que tenham definições distintas em intervalos separados. Isto se faz utilizando “.” como delimitador invisível direito, ou seja, $\backslash$right..

\begin{displaymath}
y=\left\{ \begin{array}{ll}
a & \textrm{se $d>c$}\\
b+x & \textrm{5}\\
1 & \textrm{qualquer outro valor}
\end{array}\right.
\end{displaymath}

\subparagraph{} Para formar sistemas de equações.

\begin{eqnarray}
f(x) & = & \cos x \\
f’(x) & = & -\sin x \\
\int {0}^{x} f(y)dy & = & \sin x
\end{eqnarray}

\begin{flushleft}
\textbf{Tamanho do tipo para equações:}
\end{flushleft}

\paragraph{} No modo matemático TEX seleciona o tamanho do tipo segundo o contexto. Os períndices, por exemplo, se dispõem num tipo mais pequeno. Portanto, deve-se usar $\backslash$mathrm para que se mantenha ativo o mecanismo de troca do tamanho. Porém, cautela, já que $\backslash$mathrm só funcionará bem
com coisas pequenas. 

\begin{equation}
2^\textrm{o} \qquad 2^\mathrm{o}
\end{equation}

\begin{flushleft}
\textbf{Descrevendo variáveis:}
\end{flushleft}

\paragraph{} Para algumas das suas equações podesse desejar anexar uma seção onde sejam descritas as variáveis utilizadas. O seguinte exemplo poderá ser de ajuda para essa operação:

\begin{displaymath}
a^2+b^2=c^2
\end{displaymath}

{\settowidth{\parindent}{onde:\ } \makebox[3pt][r]
{onde:\ }$a$, $b$ são os adjuntos do ângulo reto de um triângulo retângulo. $c$ é a hipotenusa do triângulo}

\begin{flushleft}
\textbf{Teorema, leis...}
\end{flushleft}

\paragraph{} Quando s˜ao escritos documentos matemáticos, s˜ao empregados lemas, definições, axiomas e estruturas similares. 

\newtheorem{mur} {Lei de Murphy}[section]
\begin{mur} Se alguma coisa pode dar errado, dará.
\end{mur}

\begin{flushleft}
\textbf{Outros exemplos de fórmulas matemáticas:}
\end{flushleft}

\subparagraph{}
\begin{displaymath}
{F}(x,y)=0\quad\mathrm{and}
\quad \left| \begin{array}{ccc}
F {xx}´´ & F {xy}´´ & F {x}´ \\
F {yx}´ & F {yy}´ & F {y}´ \\
F {x}´ & F {y}´ & 0
\end{array} \right| = 0
\end{displaymath}

\subparagraph{} 

\[ \left(\begin{array}{cc}
1 & 2 \\ 0 & 1
\end{array}\right)
\left(\begin{array}{cc}
2 & 0 \\ 1 & 3
\end{array}\right)
=
\left(\begin{array}{cc}
4 & 6 \\ 1 & 3
\end{array}\right) \]

\begin{flushleft}
\textbf{O comando includegraphics:}
\end{flushleft}

\paragraph{}
O Tex  traz o pacote graphicx para a inserção de imagens no formato EPS. Isto pode ser feito com o comando $\backslash$includegraphics cuja sintaxe é:


\subparagraph{}
\begin{figure}[h]
\begin{center}
\includegraphics[width=5cm]{LogoUFT.png}
\end{center}
\caption{Logo da UFT}\label{fig:02}
\end{figure}


\subparagraph{}
\begin{figure}[h]
\begin{center}
\includegraphics[width=10cm]{Gráfico Aula 2.png}
\end{center}
\caption{Gráfico}\label{fig:02}
\end{figure}




\end{document}



