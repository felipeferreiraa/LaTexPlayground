\documentclass[a4paper, 12pt]{article}
\usepackage[top=2cm, bottom=2cm, left=2.5cm, right=2.5cm]{geometry}
\usepackage[utf8]{inputenc}
\usepackage{amsmath, amsfonts, amssymb,}
\usepackage{graphicx}
\usepackage[portuguese]{babel}
\usepackage{float}
\usepackage{tikz}
\usepackage{tabu}
\usetikzlibrary{arrows}
\usepackage{setspace}
\usepackage{cases}


\begin{document}

\begin{center}

\textbf{Universidade Federal do Tocantins}
\\
\textbf{Ciências Econômicas } \onehalfspacing 


\end{center}

\begin{flushleft}
\textbf{Curso:} Introdução ao Latex. 
\\
\textbf{Tempo de curso:} 3 aulas.
\\
\textbf{Instrutores:} Felipe Ferreira e Victor Gabriel Caldas \onehalfspacing

\begin{center}

\textbf{Plano de Curso}
\onehalfspacing
\end{center}

\textbf{Objetivo} \onehalfspacing

Apresentar e trabalhar com a linguagem de marcação Tex, demonstrando suas funções e utilidades, compreendendo sua forma de escrita, matematica e gráfica. Analisando os problemas mais usuais da linguagem e incentivar os alunos a escreverem na linguagem Tex. \singlespacing


\textbf{Conteúdo Programático} \singlespacing

Aula \textbf{01} - Introdução ao Tex e a escrita. \singlespacing

   - O formato do documento.
\\
   - Formato da folha. 
\\
   - Compondo texto.
\\
   - Caracteres especiais e símbolos.
   \\
   - Espaçamento.
   \\
   - Ambientes.
   \\
   - Elementos flutuantes.
   \\
   - Anexando novas instruções e ambientes. \singlespacing
   
Aula \textbf{02} - Introdução ao conteúdo Matemático. \singlespacing

  -  Introdução ao modo matemático.
  \\
  -  Agrupando no modo matemático.
  \\
  - Elementos das fórmulas matemáticas.
  \\
  - Matrizes.
  \\
  - Tamanho do tipo para equações.
  \\
  - Descrevendo variáveis.
  \\
  - Teoremas, leis...
  \\
  - Outros exemplos de fórmulas matemáticas.
  \\
  - Lista de símbolos matemáticos.
  \\
  - Inclusão de imagens e gráficos. \singlespacing
  
Aula \textbf{03} - Introdução a Lista de tabelas e aos gráficos. \singlespacing

  - Exemplo para um artigo em português.
  \\
  - Classes de documentos.
  \\
  - Pacotes.
  \\
  - Tipos de letras
  \\
  - Tipos de textos sublinhados.
  \\
  - Construções Matemáticas.
  \\
  - Operadores.
  \\
  - Gráficos.
  \\
  - Introdução ao Tikz.
  \\
  - Construção de gráficos. \singlespacing
  
\textbf{Avaliação} 

Solicitar uma criação de um documento ao final do curso, demonstrando algumas habilidades no Tex. \singlespacing


\textbf{Local de encontro:}
Zoom, Skype ou GoToMeeting (Poderá sofrer alterações, conforme for a demanda pelo curso) \singlespacing

\textbf{Datas das aulas}
\\
Aula 01 - 19/05
\\
Aula 02 - 20/05
\\
Aula 03 - 21/05 \singlespacing
  \begin{center}
 
  
  \textbf{Referências:} \singlespacing
  Curso de Latex - Gilberto Souto.
\\
 $www.uft.edu.br/engambiental/prof/catalunha/arquivos/latex/latex_GilbertoSouto.pdf$  \singlespacing
  Graphing in LATEX using PGF and TikZ - Lindsey-Kay Lauderdale, Mathew R. Gluck. Department of Mathematics, University of Florida. \singlespacing
 
  A very minimal introduction to TikZ - Jacques Crémer. Toulouse School of Economics \singlespacing
  \end{center}
\end{flushleft}

\end{document}