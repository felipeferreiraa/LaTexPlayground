\documentclass[a4paper, 12pt]{article} %Tipo de documento.
\usepackage[top=2cm, bottom=2cm, left=2.5cm, right=2.5cm]{geometry} %Pacote para Margens
\usepackage[utf8]{inputenc} % Pacote para permitir a utilização de acentos.
\usepackage{amsmath, amsfonts, amssymb, dsfont} % % Pacote matemático
\usepackage{graphicx} % Pacote inserir imagens.
\usepackage[portuguese]{babel} % Pacote para usar diversos idiomas
\usepackage{float} %Para inserir textos nas imagens
\usepackage{tikz} % Pacote para gráficos Tikz
\usepackage{tabu} %Pacote para tabular
\usetikzlibrary{arrows} % Pacote para o Tikz
\usepackage{setspace} %Pacote para espaçamento entre linhas
\usepackage{cases} % Para definir ordens ao modelo matemático
\usepackage[normalem]{ulem} % Pacote para usar o uuline (texto sublinhado)
\usepackage{multirow} % Pacote para criação de Tabular.
\usepackage{tikz-3dplot}

\begin{document}

\begin{center}
\textbf{UNIVERSIDADE FEDERAL DO TOCANTINS\\
	CAMPUS UNIVERSITÁRIO DE PALMAS\\
	COORDENAÇÃO DE ECONOMIA}
\end{center}

\textbf{Prof: Fernando Jorge Fonseca Neves }
\singlespacing
\textbf{Aluno: Felipe Ferreira de Sousa}
\begin{center}
\textbf{Atividade 1: Ciclos do Capital}
\end{center}

\par \textbf{Capítulo 1: circulação de mercadorias e circulação de capital}
\vspace{0.5cm}
\par \textbf{1)} O estudo do movimento do capital é realizado por Marx no volume 3 da sua obra “O Capital”. De acordo com o método deste autor, estudar a produção do capital, apenas, não fornece subsídios suficientes para compreender o modo de produção capitalista. O fenômeno “capital”, revelado por meio do excedente de valor que resulta da produção capitalista (mais-valia), existe pela sua
continuidade, ou seja, pela sua reprodução contínua – o que motiva o estudo do movimento do capital. O movimento do valor-capital, \textbf{K} ou \textbf{VK} (conteúdo), ocorre por meio do seu percurso através das várias
formas (três) que lhe dá existência específica.

\subparagraph{} \textbf{A)} Enumere as três formas de existência do capital e procure distinguir o conteúdo (K ou VK) das
suas formas.

\subparagraph{} \textbf{B)} Aponte quais as formas típicas da esfera da produção e quais as formas da esfera da circulação.
O que distingue, do ponto de vista do valor, as formas do capital em cada um desses dois lócus
de existência?

\subparagraph{} \textbf{C)} A partir da figura do capital industrial, mostre cada forma do capital e os respectivos lócus de
existência. Desenhe, separadamente, a figura dos três ciclos do capital. Defina ciclo do capital
\vspace{0.5cm}
\par \textbf{Capítulo 3: o ciclo do capital-dinheiro}
\vspace{0.5cm}
\par \textbf{2)} Da figura do ciclo do capital-dinheiro, pode-se destacar três estádios (ou estágios) distintos: (i) o
estágio D – M; (ii) o estágio ... P ... M’; e (iii) o estágio M’ – D’

\subparagraph{} \textbf{A)} Se o valor-capital, K ou VK, tem a sua existência caracterizada por ser valor em progressão ou
valor que cresce pelo movimento, utilize a sequência dos diferentes estágios do ciclo do
capital-dinheiro para analisar como, iniciando-se na forma D ou KD, lhe é imposto que mude, sucessivamente, de forma, até que regresse novamente á forma original de D ou KD.

\subparagraph{} \textbf{B)} O que a análise do ciclo do capital dinheiro permite concluir?
\vspace{0.5cm}
\par \textbf{Capítulo 4: o ciclo do capital-produtivo}
\vspace{0.5cm}


\textbf{3)} Da figura do ciclo do capital-produtivo pode-se destacar dois estágios distintos: (i) o estágio ... P ...
M’; e (ii) o estágio M’ – D’ – M (Mp + Ft).

\subparagraph{} \textbf{A)} Caracterize cada um desses estágios do ponto de vista do lócus onde ocorrem (ou esfera onde
o capital está localizado) e descreva o comportamento do capital na sequência das formas do
ciclo do capital-produtivo até que este se complete.

\subparagraph{} \textbf{B)} Ao contrário do ciclo do capital-dinheiro, no ciclo do capital-produtivo interessa observar o
destino dado a D’ após M’ – D’, pois, uma vez que a circulação interrompe a produção,
interessa saber as circunstâncias da retomada da produção. Esta retomada pode ser de
reprodução simples ou de reprodução ampliada. Desenhe uma versão do ciclo do capital-
produtivo que faça a distinção entre estas duas formas de reprodução. Explique cada uma,
diferenciando-as.

\subparagraph{} \textbf{C)} O que a análise do ciclo do capital-produtivo permite concluir?

\vspace{0.5cm}

\par \textbf{Capítulo 5: o ciclo do capital-mercadoria}
\vspace{0.5cm}

\par \textbf{4)} Da figura do ciclo do capital-mercadoria pode-se destacar três estágios distintos: (i) M’ – D’; (ii) D’ – M
(Mp + Ft); e (iii) (Mp + Ft) ... P .... A forma M’ impõe que o ciclo compreenda obrigatoriamente uma
circulação completa, com seus dois estágios opostos, M’ – D’ (venda) e D’ – M (compra). Pela forma,
apresenta-se como movimento simples de circulação de mercadorias, M – D – M, cujo propósito é o
valor-de-uso. Pelo conteúdo, é parte integrante do movimento de circulação do capital devido à
natureza das mercadorias compradas, Mp + Ft, que são elementos do capital produtivo (marcam a
interrupção da circulação) cujo valor-de-uso é produzir mais-valia.

\subparagraph{} \textbf{A)} Desenhe a figura do ciclo do capital-mercadoria, diferenciando, em cada forma do capital, o
valor primitivo e a mais-valia; faça a distinção, na figura, das hipóteses de reprodução simples e
de reprodução ampliada por meio das alternativas de consumo do valor excedente por parte
do capitalista.

\subparagraph{} \textbf{B)} A forma mercadoria aparece no início, no meio e no fim do ciclo. O M’ final, por ser resultado
da função produtiva de ... P ..., tem a sua origem conhecida. O M que aparece no meio (Mp +
Ft) vem de fora do ciclo de referência, é resultado de outros ciclos, e deve, por isso, ser objeto
de investigação. Explique, e discuta o fato do ciclo do capital-mercadoria pressupor a existência
simultânea de outros processos produtivos.

\subparagraph{} \textbf{C)} Da constituição ou reconstituição das condições de produção, o ciclo do capital-mercadoria
revela interações e conexões entre produtores capitalistas que produzem meios de consumo
final (Mc) e aqueles que produzem meios de produção (Mp); entre produtores capitalistas que
produzem meios de produção entre si; e entre produtores capitalistas, em geral, e produtores
de força de trabalho (ou trabalhadores). Ao ajuntarem-se todos os produtores de meios de
consumo, constitui-se o Departamento 2; e ao ajuntarem-se todos os produtores de meios de
produção, constitui-se o Departamento 1. Sabe-se que o Departamento 1 produz todos os
meios de produção de que a sociedade precisa (ambos os departamentos) para produzir. Além
disso, sabe-se que todos os trabalhadores (ambos os setores) repõem a sua capacidade de
trabalho com os meios de consumo produzidos pelo Departamento 2, que também fornece
esses meios para o consumo dos capitalistas. Desenhe um diagrama do ciclo do capitalmercadoria onde figuram os dois departamentos, mostrando, por meio de setas, as várias
interações descritas. Suponha a reprodução simples do capital.

\subparagraph{} \textbf{D)} O que a análise do ciclo do capital-mercadoria permite mostrar?

\vspace{0.5cm}

\par \textbf{Capítulo 6: os três ciclos do capital em conjunto}

\vspace{0.5cm}

\par \textbf{5)} Apresente as três figuras como pode aparecer o ciclo do capital [ciclo do capital dinheiro (KD), ciclo
do capital produtivo (KP) e ciclo do capital mercadoria (KM)]. Deve reparar que cada ciclo é constituído
pela sequência de fases de produção e de circulação (ou vice-versa). A fase inicial é sempre igual à fase
final, e a fase do meio (diferente daquelas duas) constitui, ao mesmo tempo, a interrupção da fase
inicial e condição para que a fase inicial se repita na fase final. A partir deste princípio (fase inicial e fase
final mostram o objetivo do ciclo e a fase do meio é condição para o seu cumprimento), proponha
observações relevantes sobre cada ciclo, distinguindo-os.

\vspace{0.5cm}

\begin{center}
\textbf{Respostas:}
\end{center}

\par \section{Capítulo 1: circulação de mercadorias e circulação de capital}
\vspace{0.5cm}

\par \textbf{1)}


\subparagraph{} \textbf{A)} As três formas que Marx define a existência do capitão são as seguitnes \textbf{capital-dinheiro(KD ou D), capital produtivo (KP ou P) e capital-mercadoria (KM ou M)}. O capital-dinheiro é o ciclo do capital que lidamos com o dinheiro com a circulação para demonstração iremos fazer o ciclo da seguinte forma:
\begin{equation}
D-M... \ P \ ... M'-D'
\end{equation}
Veja, no capital-dinheiro (KD)) temos a noção de se trata dessa circulação em dois momentos, no primeiro D-M quando se usa dinheiro para comprar MP e FT e no último setor quando temos a noção mercadoria já com a mais-valia sendo vendida para conseguir o dinheiro e consequentemente o lucro desta mesma.
\\
O capital produtivo (KP) é a quebra dessa circulação inserindo o fator produtivo no movimento do capital, o primeiro estágio era da movimentação D-M, agora no produtivo teremos o D-M ... P. O \textbf{KP} é ´constituído de capital produtivo e não produtivo, em tese, nesse processo é onde temos o trabalho sendo efetuado ao total vapor, criando produtos e realizando a constituição física da FT + MP.
\\
Por último temos o capital-mercadoria (KM) é a volta do capital em circulação com as taxas de mais-valia já acrescentadas a mercadoria, nessa etapa o capitalista vai atrás de realizar o seu lucro em cima das taxas de mais-valia usando, um exemplo disso é M' = M + $\mu$. Nessa etapa, sabemos que o \textbf{KM} será usado para realizar novamente os processos.

\subparagraph{} \textbf{B)} A forma tipica da esfera da produção é P ... M' - D' - M (Este contém MP E FT) ... P. Inicia-se no ponto P, onde o capital produtivo é iniciado, demonstrando que todos os fatores de produção estão em pleno acordo para a sua produção. O interessante é que nessa fase de capital produtivo, temos que o ciclo inicia por P e termina por P, ou seja, o capitalista realiza todo esse processo novamente para realizar esse processo inúmeras vezes. 
\\
Bom, agora temos dois processos na esfera da circulação que é KD (Capital-dinheiro) e KM (Capital-mercadoria), o capital-dinheiro é bem conhecido como D-M, quando o capitalista compra os meios de produção e a força de trabalho para a indústria, inicia-se o lócus da circulação, quando o capitalista está comprando os meios para produção e na última parte, temos o processo que o capitalista vende a mercadoria (com mais-valia já incluída) para realizar o seu lucro monetário.
\\
A constituição de valor no KP se dá na produção, ou seja, o ato de gerar a constituição do MP e da FT, em coexistência o capitalista gera a agregação dessas forças a mercadoria, gerando o seu valor. Agora no KM e KD que estão na esfera da circulação, o que gera valor nessa etapa é a circulação de dinheiro e mercadoria, mas a prior mesmo é no capital-dinheiro, porque é a existência do valor-de-uso que é o dinheiro.

\subparagraph{} \textbf{C)} O capital industrial é dá seguinte forma:
\begin{equation}
 MP + FT \ ... \ P \ ... \ M' - D' - M - MP+FT \ ... \ P \ ... 
\end{equation}

Sabemos que MP+FT ... P ... é o KP (capital industrial), M' é KM (capital-mercadoria), D' é KD (capital-dinheiro) e retornamos ao KP M - MP+FT ... P ... . Outro ponto é que o KP está na esfera da produção, KM + KD está na esfera da circulação. \\
No primeiro estágio, temos ... P ... M', sabemos que é o processo do valor-de-uso dos meios de mão de obra para gerar a mercadoria com mais-valia, no segundo estágio temos a circulação M' - D' - M = MP+FT, esse é um estádio curioso para a nossa avaliação, pois é nele, que temos a valorização do M', que é constituído de M = M+ $\mu$, já o D' = D + d, é um processo no qual o dinheiro define se está ocorrendo um processo de reprodução simples ou ampliada. Se o d tiver por destino o capital produtivo, em síntese, um capital para gerar produtividade e novos incrementos ao capital primitivo isso será uma forma de ampliar o funcionamento. Agora se o d for para consumo pessoal do capitalista ao todo, será uma forma de reprodução simples.
\\
O ciclo do capital produtivo é da seguinte forma: seguindo a função de KI, produção e reprodução, depois temos o processo de acumulação como capitalização de mais-valia e por final teremos a função de D', se tem a função de ser uma reprodução simples ou ampliada.
\vspace{0.5cm}

\par \section{Capítulo 3: o ciclo do capital-dinheiro}
\vspace{0.5cm}
\par \textbf{2)}
\subparagraph{} \textbf{A)} O capital-dinheiro (KD) é definido da seguinte maneira: 

\begin{equation}
D-M... \ P \ ... M'-D'
\end{equation}
O primeiro estágio do capital-dinheiro é o D-M (compra), como já sabemos é um estádio de circulação, nele o capitalista utiliza-se do seu capital para iniciar os processos produtivos ou melhor, o capitalista adquire M = Meios de produção + Força de trabalho. Ele irá se diferir do terceiro processo de circulação, que iremos expor logo a seguir. Neste processo, o capitalista é o "escravo" do seu capital, pois ele nada pode fazer com esse dinheiro a não ser que o KP retorne dando lucro.
\\
O segundo estádio do capital- dinheiro é o ... P ... M', neste estádio estamos quebrando a circulação para a entrada do capital produtivo, aqui iremos ter a esfera mais tangível de todas - produção de mercadorias. O M' é composto de mais-valia oriundas dos processos de MP e FT, compostos de elementos objetivos e subjetivos.
\\
O terceiro estágio do capital-dinheiro é o M'-D', neste estádio estamos lidando com a venda de mercadorias, que está acrescentada de excedentes da produção, para ilustração: M' = M + $\mu$ e D' = D + d. Ou seja, M'-D' é a definição da venda com mais-valia em demonstração, ou seja, valor-capital (VK) é uma situação diferente, pois M'-D' retorna ao processo inicial de circulação, demonstrando que o ciclo é um "ad eternum".

\subparagraph{} \textbf{B)} Algumas conclusões podemos tomar destes processos, um inicial é que é um ciclo que KD é um processo que produz mais-valia pelo seu dinheiro, que este processo tem por objetivo o acréscimo de valor e que este ciclo irá gerar mais dinheiro ao capitalista - real objetivo. Outra conclusão é que esse ciclo é ideal para se estudar de forma individual e demonstrar suas funções, sem que ocorra inúmeras movimentações semelhantes.

\vspace{0.5cm}

\par \section{Capítulo 4: o ciclo do capital-produtivo}
\vspace{0.5cm}

\par \textbf{3)}

\subparagraph{} \textbf{A)} o primeiro estágio ... P...M' é o estádio inicial da produção, é neste estádio que será acrescentado a taxa de mais-valia. Ou seja, o estágio P (produção) gerá mais taxa de mais-valia M'. O P é a junção de meios de produção e força de trabalho. O segundo estádio M' - D' - M (MP + FT) é o modo circulação deste processo, constitui-se de M' = M + $\mu$ e D' = D + d, neste processo que também ocorre a venda e a transformação de dinheiro. Neste segundo estágio temos dois fatores novos que são reprodução simples e reprodução ampliada, que tem a origem vindo do d.

\subparagraph{} \textbf{B)} A reprodução simples e ampliada tem origem na segunda fase do ciclo do KP que é no estádio M' - D' - M(MP + FT), mais precisamente na fase do D' = D + d. O que determina se será uma RS ou RA é a finalidade do d, então, se houver reprodução simples, significa que o d foi usada para consumo pessoal do capitalista ou melhor, consumo improdutivo é o caso que a mais-valia é gerada para consumo pessoal do capitalista.
\\
Já a reprodução ampliada é quando o capitalista usa o seu capital para consumo produtivo, quando uma fração da mais-valia é gerada para uma nova produção para gerar outras novas produções.

\subparagraph{} \textbf{C)} As conclusões que o KP nos permite é que a função do KI é provado, primeiro, pela produção e reprodução do ciclo, depois, houve um processo de acumulação de como capitalização e por último a função do D' e se ocorre uma separação para capital produtivo ou improdutivo. É um ciclo perfeito para individualidades e funções técnicas ao capital.

\vspace{0.5cm}

\par \section{Capítulo 5: o ciclo do capital-mercadoria}
\vspace{0.5cm}

\subparagraph{} \textbf{A)} o ciclo do capital-mercadoria é da seguinte forma: 
\begin{equation}
M' - D' - M (MP + FT) ... P ... M'''
\end{equation}
\begin{center}
ou
\end{center} 
\begin{equation}
C ... P ... M'''
\end{equation}

O ciclo da mercadoria é iniciado em M', capital sob forma de mercadoria, porém, lembrando que esse ciclo inicia a partir do P, que são os processos industriais e de unificação de MP + FT, gerando mais-valia, até retornar em M para compra de MP + FT, esse ciclo de M'- é o KM (capital-mercadoria). O KD (capital-dinheiro) inicia-se em D' e  M(MP+FT) ... P ... é o KP(capital produtivo) e novamente retorna ao M', capital mercadoria.
Sabemos que a circulação acontece em dois movimentos do capital que é no KD e KM, nesse ciclo temos a noção de que a mais-valia inicia-se no M'  e o valor primitivo volta na parte de compras de meio de produção para a indústria. \\
A reprodução simples acontece pelo fato de quando a D' = D + d, o capitalista tem uma finalidade ao d, se o objetivo estiver em satisfazer consumos pessoais, é o que chamamos de reprodução simples, porém, se ele quiser reinvestir essa fatia de lucro, chamamos de reprodução ampliada, pois irá por em funcionamento o ciclo.

\subparagraph{} \textbf{B)} Capital-mercadoria ou KM, é síntese dos dois processos anteriores, pois ele demonstra como ocorre a circulação e produção no KD, que acontece com a forma que migra entre o ciclo. A KM no KD está incluído quando realiza a forma de dinheiro nesse ciclo, se analisarmos de forma minuciosa, o KM é a junção do capital produtivo e mercadoria, de forma branda, o KM é a demonstração do que ocorre com a transferência da mais-valia para o lucro do capitalista.

\subparagraph{} \textbf{C)} 
\begin{figure}[h]
	\begin{center}
		\caption{Desenho do ciclo de mercadorias e os departamentos. }
		\includegraphics[width=10cm]{Movimento.pdf}
	\end{center}
	\caption{Gráfico}\label{fig:02}
\end{figure}

O que ocorre nesse fluxograma é o seguinte, existem 4 setores: Setor 1: Produção de meios de produção, Setor 2: Produção de meios de produção, Setor 3: Produção de meios de consumo necessários e Setor 4: Produção de meios de consumo de luxos. O Setor 1 fornece meios de produção ao setor 3 e 4 , setor 2 fornece meios de produção ao setor 1, 4 e para o próprio setor. O Setor 3 oferece meios de consumo ao capitalistas e operários e por último, o setor 4 oferece meios de consumo para todos os proprietários de capital nos 4 setores demonstrados. Se formos pensar na reprodução simples deste processo, é de analisar que o ciclo e a velocidade de todo esse processo industrial será menor e reduzido, afinal, o capitalista irá colocar cada vez menos capital nos setores I e II, e só irá consumir de III e IV (neste caso, apenas ele), então, é de se pensar que o setor dos operários será bem reduzido e como também a disponibilidade de empregos para este setor.

\subparagraph{} \textbf{D)} O que ciclo-mercadoria permite afirmar são o entrelaçamento dos ciclos, a produção capitalista global, a distribuição da produção social em fundo do consumo individual e fundo de produção, o consumo total seja produtivo ou improdutivo e demonstra a origem e o fim das mercadorias na sociedade.


\vspace{0.5cm}

\par \section{Capítulo 6: os três ciclos do capital em conjunto}
\vspace{0.5cm}


\subparagraph{} \textbf{5 -} Os três ciclos são da seguinte maneira: 

\begin{equation}
KD = D \ - \ M \ ... \ P \ ... \ M' \ - \ D'
$$\\$$
KM = M' \ - \ D' - \ M(MP + FT) \ ... \ P \ ... \ M'''´
$$\\$$
KP = M(MP + FT) \ ...\ P\ ... \ M' \ - \ D' - \ M(MP + FT) \ ... \ P \ ...´
\end{equation}

Podemos fazer algumas distinções destes três ciclos, por exemplo, KD, a produção é interrupção da circulação. No KP, a circulação é interrupção da produção e KM, a circulação abre o ciclo. O KD a circulação aparece no D - M - D e no KM/KP, a circulação aparece como M - D - M. 
\\
O KD tem algumas considerações, por exemplo, D - M é a primeira transformação de KD e KP, no novo ciclo, desaparece a forma como foi produzido, D' torna-se D e P' converte-se em P'.
\\
O KP tem outras particularidades, como por exemplo, a D-M é a volta de KM e KP, de onde tinha surgido, a produção é a uma pré-condição ao ciclo, a forma final de P é a transformação de D. É também uma mostra da capitalização da mais-valia e também, KP é a forma clássica que a economia observa-se os movimentos industriais.
\\
KM possui características semelhantes ao KP, a circulação também volta como M-D-M, D-M é a condição inicial para a formação da mais-valia M', outro detalhe interessante é que nesse ciclo temos a noção de como é a realizado o processo produtivo, como também entende-se que a ocorrência de outros ciclos simultâneos, e que a sua forma final M' é o produto do seu ciclo, outro ponto, é que neste ciclo a demonstração da reprodução ampliada e simples, demonstrando o destino do consumo do capitalista. E por fim, serve de base para o quadro econômico de Quesney e para a matriz insumo produto.

\end{document}