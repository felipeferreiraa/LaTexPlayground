\documentclass[a4paper, 12pt]{article} %Tipo de documento.
\usepackage[top=2cm, bottom=2cm, left=2.5cm, right=2.5cm]{geometry} %Pacote para Margens
\usepackage[utf8]{inputenc} % Pacote para permitir a utilização de acentos.
\usepackage{amsmath, amsfonts, amssymb, dsfont} % % Pacote matemático
\usepackage{graphicx} % Pacote inserir imagens.
\usepackage[portuguese]{babel} % Pacote para usar diversos idiomas
\usepackage{float} %Para inserir textos nas imagens
\usepackage{tikz} % Pacote para gráficos Tikz
\usepackage{tabu} %Pacote para tabular
\usetikzlibrary{arrows} % Pacote para o Tikz
\usepackage{setspace} %Pacote para espaçamento entre linhas
\usepackage{cases} % Para definir ordens ao modelo matemático
\usepackage[normalem]{ulem} % Pacote para usar o uuline (texto sublinhado)
\usepackage{multirow} % Pacote para criação de Tabular.
\usepackage{tikz-3dplot}

\begin{document}

\begin{center}
\textbf{UNIVERSIDADE FEDERAL DO TOCANTINS\\
	CAMPUS UNIVERSITÁRIO DE PALMAS\\
	COORDENAÇÃO DE ECONOMIA}
\end{center}

\textbf{Prof: Fernando Jorge Fonseca Neves }
\singlespacing
\textbf{Aluno: Felipe Ferreira de Sousa}
\begin{center}
\textbf{Atividade 4: Relação entre Taxa de Lucro, Composição do Capital Adiantado e Taxa de Mais-valia}
\end{center}
\vspace{0.5cm}


\par \textbf{1-} Suponha uma economia composta por dois departamentos destinados a produzir mercadorias
durante um ano. O Departamento I é o departamento produtor de meios de produção (Mp); e o
Departamento II é o departamento produtor de meios de consumo (Mc). O Departamento I e o
Departamento II são constituídos, respectivamente, pelos seguintes capitais sociais [Nota: a composição
orgânica do capital, c/v (medida tecnológica estrutural), e a taxa de mais-valia, m/v (medida da
distribuição de renda), permanecem as mesmas]:

\vspace{0.5cm}
\begin{center}
\begin{tabular}{|c|c|}
	\hline \multicolumn{2}{|c|} {\text { CAPITAL SOCIAL }} \\
	\hline \text { Departamento I } & \text { Departamento II } \\
	\hline 1) 4.000c + 1.000v  (c/v=4) & 11) 2.000c+500v (c/v=4) \\
	\hline
\end{tabular}
\end{center}

\subparagraph{} \textbf{A)} Se a taxa de mais-valia é m’ = m/v = 100\%, mostre a composição do valor da mercadoria (M = c+ v + m) produzida ao longo de um ano por cada um dos departamentos; 

\subparagraph{} \textbf{B)} Mostre a oferta total da economia decorrente das atividades de cada um dos departamentos, e
mostre a procura total, distinguindo cada agente que realiza cada tipo de procura;

\subparagraph{} \textbf{C)} Supondo a Reprodução Simples do Capital, a produção deste primeiro ano deixa pronta a
condição de reprodução para o segundo ano? Como? Deduza essa condição;

\subparagraph{} \textbf{D)} Supondo a Reprodução Ampliada do Capital: se os capitalistas do Departamento I dedicam, ao
final do primeiro ano, metade da sua mais-valia para efeito de reinvestimento para o segundo
ano, que consequências haveria para a reprodução do capital total da sociedade neste segundo
ano, considerando as relações entre oferta e procura?

\subparagraph{} \textbf{E)} A que conclusões pode chegar sobre a Reprodução Ampliada do Capital no que se refere às
questões de proporção da produção nos dois departamentos?

\vspace{0.5cm}

\par \textbf{2-}  Suponha uma economia composta por dois departamentos destinados a produzir mercadorias
durante dois anos. O Departamento I é o departamento produtor de meios de produção (Mp); e o
Departamento II é o departamento produtor de meios de consumo (Mc). O Departamento I e o
Departamento II são constituídos, respectivamente, pelos seguintes capitais sociais:


\vspace{0.5cm}
\begin{center}
	\begin{tabular}{|c|c|}
		\hline \multicolumn{2}{|c|} {\text { CAPITAL SOCIAL }} \\
		\hline \text { Departamento I } & \text { Departamento II } \\
		\hline I) 4.000c + 1.000v  (c/v=4) & II)  1.500c + 750v (c/v = 2) \\
		\hline
	\end{tabular}
\end{center}

\subparagraph{} \textbf{A)} Se a taxa de mais-valia é 100\%, mostre a composição do valor da mercadoria (M = c + v + m)
produzida ao longo do primeiro ano por cada um dos departamentos;

\subparagraph{} \textbf{B)} Mostre a oferta total da economia decorrente das atividades de cada um dos departamentos, e
mostre a procura total, distinguindo cada agente que realiza cada tipo de procura;

\subparagraph{} \textbf{C)} Supondo a Reprodução Ampliada do Capital: se os capitalistas do Departamento I dedicam, ao
final do primeiro ano, metade da sua mais-valia para efeito de reinvestimento para o segundo
ano, como seriam as condições para o processo de reprodução social (nos dois departamentos
em conjunto) se os capitalistas do Departamento II aproveitarem todo o excedente de meios
de produção originários do departamento I para procederem à sua própria reprodução
ampliada?

\subparagraph{} \textbf{D)} Neste caso, as condições de expansão da produção ocorreriam em equilíbrio entre oferta e
demanda nos dois departamentos. Mostre como.

\vspace{0.5cm}

\par \textbf{3)}  Discorra sobre as considerações finais do texto a respeito da reprodução ampliada do capital social.

\vspace{0.5cm}

\begin{center}
\textbf{Respostas:}
\end{center}

\begin{center}
\section{A Reprodução do Capital}
\end{center}

\vspace{0.5cm}

\par \textbf{1)}

\subparagraph{} \textbf{A)} Se a taxa de mais-valia for de 100\%, logo teremos o seguinte:
\\


\begin{center}
\begin{tabular}{cccccccc}

	D-I) & 4.000c & + & 1.000v & + & 1.000m &=& 6.000 (VBP-I) \\
	D-II) & 2.000c & + & 500v & + & 500m &=& 3.000 (VBP-II)
\end{tabular}
\end{center}

Essa será a composição \textbf{M = c + v + m }.

\subparagraph{} \textbf{B)} Já a oferta total da Economia será assim: 

\begin{center}
	\begin{tabular}{cccccccc}

		& $\underline{MP}$ & & $\underline{MC}$ & & $\underline{MC}$ \\
		D-I) & 4.000c & + & 1.000v & + & 1.000m &=& 6.000 (VBP-I) \\
		D-II) & 2.000c & + & 500v & + & 500m &=& 3.000 (VBP-II) \\
		&  $\overline{6.000}$ & & $\overline{1.500}$ & & $\overline{1.500}$ & & $\overline{9.000}$
		
	\end{tabular}
\end{center}

Ou seja, sabemos que o investimento (capital social) será de 5.000 e o seu produto mercadoria de 6.000 (4.000c + 1.000v + 1.000m) referente ao departamento I. No departamento II, temos o capital social de 2.500 (2.000c + 500v) que irá gerar um produto mercadoria de 3.000 (2.000c + 500v + 500m).
\\
A relação entre esses dois departamentos ocorram da seguinte maneira, trabalhadores e capitalistas de D-I, dispondo de renda equivalente a R\$ 2.000 = 1.000v + 1.000m, compram MC de que necessitem junto a D-II, permitindo que este realiza todo o valor de sua produção não consumida dentro do sistema. Os capitalistas D-II, recebendo esta soma monetária de R\$ 2.000 pela venda realizada, compram os MP que necessitam de 2.000c, ao mesmo tempo, esta venda permite que D-I realize a parte do valor de seu produto não consumido dentro de si mesmo.
\\
Podemos pensar, se os capitalistas desejam do D-I 3/5 da sua mais-valia para adquirir MC necessário e 2/5 para adquirir MC de luxo, tal que 1.000m = (600 + 400)m

\begin{equation}
\textit{D-I)}\ 4.000c + 1.000v + 1.000m = 6.000\textit{(em meios de produção)}
$$\\$$
\textit{D-IIA)}\ 1.600c + 400v + 400m = 2.400\textit{(em meios de consumo necessário)}
$$\\$$
\textit{D-IIB)}\ 400c + 100v + 100m = 600\textit{(em meios de consumo de luxo)}
\end{equation}

\subparagraph{} \textbf{C)} Pela condição de equilíbrio, haverá condições ideais para que o sistema continue de forma harmoniosa.

\begin{center}
1) I(c + v + m) = Ic + IIc
\\
2) II(c + v + m) = I(v + m) + II(v + m)
\end{center}

O dinheiro recebido pelos capitalistas da D-II, não é despendido em compras do D-I, para ser mais claro isso é o que corresponde a amortização do capital. Então, irá ocorrer uma compensação, pois, se uma parte é poupada, haverá outra que será utilizada na compra de MP e a outra que é vindo de um fundo de amortização de outros departamentos em tempos distintos. Ou seja, I(v + m) = IIc, que é portanto, a condição de equilíbrio para que o sistema funcione de forma correta.

\subparagraph{} \textbf{D)} Lembrando que a reprodução ampliada significa a reprodução das condições necessárias ao processo produtivo em maior escala. Isto é, uma parte do produto é destinado a incrementar o novo capital nas suas partes constantes e variáveis, isso relembrando ao nosso esquema de departamento. Se pensarmos que o capitalista irá destinar 50\% da sua mais-valia como reinvestimento, ou seja, 500 unidades.
\\
Assim, mantendo as hipóteses simplificadoras em relação a composição orgânica, iremos ter uma acréscimo nos meios de produção, ou seja, dos 500 gerados, 400 serão em MP e 100 em MC. Sendo assim, (4.000 + 400)c + (1.000 + 100)v + 500m - (fundo de investimento). Entretanto, a produção do D-I é, em valor igual a 6.000 unidades e 4.000 unidades são ai mesma consumida, mas devemos que D-II depende do D-I, ou seja, o D-II só irá produzir 1.600 unidades e se o D-I precisar de 2.000 unidades, então, a acumulação de D-I impossibilitaria o D-II.
\\
Então, teremos o problema de superprodução de D-II, um exemplo a seguir:

\begin{equation}
c_{I} + v_{I} + m_{I} > c_{I} + c_{II}
$$\\$$
4.400c + 1.100v + 1.100m > 4.400c + 1.600c
$$\\$$
6.600 > 6.000
\end{equation}

\subparagraph{} \textbf{E)} Podemos concluir que nestes dois departamentos a reprodução ampliada é um problema sério pois não tende ao equilíbrio. Se a reprodução ampliada acontecer nesse processo de departamentos o D-I terá um investimento maior que o D-II, logo, o D-II vai ter um excesso de oferta.

\vspace{0.5cm}

\par \textbf{2)} 

\subparagraph{} \textbf{A)} Se a taxa de mais-valia for de 100\%, logo teremos o seguinte:

\begin{center}
	\begin{tabular}{cccccccc}
		
		D-I) & 4.000c & + & 1.000v & + & 1.000m &=& 6.000 (VBP-I) \\
		D-II) & 1.500c & + & 750v & + & 750m &=& 3.000 (VBP-II)
	\end{tabular}
\end{center}

Essa será a composição \textbf{M = c + v + m }.

\subparagraph{} \textbf{B)} Já a oferta total da Economia será assim: 

\begin{center}
	\begin{tabular}{cccccccc}
		
		& $\underline{MP}$ & & $\underline{MC}$ & & $\underline{MC}$ \\
		D-I) & 4.000c & + & 1.000v & + & 1.000m &=& 6.000 (VBP-I) \\
		D-II) & 1.500c & + & 750v & + & 750m &=& 3.000 (VBP-II) \\
		&  $\overline{5.500}$ & & $\overline{1.750}$ & & $\overline{1.750}$ & & $\overline{9.000}$
		
	\end{tabular}
\end{center}

Marx considera o mesmo VBP (9.000) para mostrar que não é desse montante que está na origem da reprodução ampliada, logo as trocas dos departamentos serão semelhantes.

\subparagraph{} \textbf{C)} Agora se a mais-valia for de 50\%, por causa do c/v = 4. Considerando que a 1.000v, teremos 500m, sendo uma alocação da seguinte maneira 400c e 100v. Agora, se lembrarmos do D-I = (4.000 + 400)c + (1.000 + 100)v + 500m(fundo de reserva). Teremos o D-I (6.000 - 4.400) $\Rightarrow$ 1.600 unidades para vender a D-II. Agora, o D-II (1.600 - 1.500) $\Rightarrow$ 100c que o D-II terá de adquirir. Assim, o D-II só poderá adquirir apenas 50v por causa da c/v = 2. Agora, iremos imaginar o processo para um período seguinte:

\begin{center}
	\begin{tabular}{cccccccc}
		
		& $\underline{MP}$ & & $\underline{MC}$ & & $\underline{MC}$ \\
		D-I) & 4.400c & + & 1.100v & + & 1.100m &=& 6.600 (VBP-I) \\
		D-II) & 1.600c & + & 800v & + & 800m &=& 3.200 (VBP-II) \\
		&  $\overline{6.000}$ & & $\overline{1.900}$ & & $\overline{1.900}$ & & $\overline{9.800}$
		
	\end{tabular}
\end{center}
Ou seja, considerando a mais-valia de 100\% esse será os valores para o próximo período.

\subparagraph{} \textbf{D)} Para ficar em equilibrio ele irá gerar alguns fatores, no D-I temos um fundo dos capitalistas de 500m e em D-II de 600m (750-100-50). A oferta corresponde 3.000 unidades, a seguir o modelo:

\begin{equation}
\textit{D-I)} \ 6.000 = 4.000 + 400 + 1.500 + 100
$$\\$$
6.000 = 6.000
$$\\$$
\textit{D-II)} \ 3.000 = 1.000 + 100 + 500 + 750 + 600
$$\\$$
3.000 = 3.000 
\end{equation}

Então, nesse caso, as condições de expansão vão ocorrer em equilíbrio para o próximo ano.


\vspace{0.5cm}

\par \textbf{3)} O problema dos meios de circulação resolve-se de forma idêntica á da reprodução simples. Existem alguns pressupostos para tais condições sejam realizadas, um dos principais é que depende da produção industrial do período anterior. Outro pressuposto é que a dinâmica do sistema deve ser procurada no interior, nas suas contradições internas e que existem proporções rigorosamente determinadas segundo as quais o sistema é capaz de reproduzir-se novamente.
\\
Também, esse estudo nos demonstra que é impossível que a reprodução seja harmônica e proporcionada. Então, a reprodução capitalista tem uma característica impar, que é ruptura das condições de equilíbrio. 

\end{document}