\documentclass[a4paper, 12pt]{article} %Tipo de documento.
\usepackage[top=2cm, bottom=2cm, left=2.5cm, right=2.5cm]{geometry} %Pacote para Margens
\usepackage[utf8]{inputenc} % Pacote para permitir a utilização de acentos.
\usepackage{amsmath, amsfonts, amssymb, dsfont} % % Pacote matemático
\usepackage{graphicx} % Pacote inserir imagens.
\usepackage[portuguese]{babel} % Pacote para usar diversos idiomas
\usepackage{float} %Para inserir textos nas imagens
\usepackage{tikz} % Pacote para gráficos Tikz
\usepackage{tabu} %Pacote para tabular
\usetikzlibrary{arrows} % Pacote para o Tikz
\usepackage{setspace} %Pacote para espaçamento entre linhas
\usepackage{cases} % Para definir ordens ao modelo matemático
\usepackage[normalem]{ulem} % Pacote para usar o uuline (texto sublinhado)
\usepackage{multirow} % Pacote para criação de Tabular.
\usepackage{tikz-3dplot}

\begin{document}

\begin{center}
\textbf{UNIVERSIDADE FEDERAL DO TOCANTINS\\
	CAMPUS UNIVERSITÁRIO DE PALMAS\\
	COORDENAÇÃO DE ECONOMIA}
\end{center}

\textbf{Prof: Fernando Jorge Fonseca Neves }
\singlespacing
\textbf{Aluno: Felipe Ferreira de Sousa}
\begin{center}
\textbf{Atividade 2: rotação do capital}
\end{center}
\vspace{0.5cm}

\begin{center}
\textbf{Capítulo 7: a rotação do capital}
\end{center}

\par \textbf{1 -} Defina o conceito de Rotação do Capital, distinguindo-o do conceito de Circulação do Capital.
\vspace{0.5cm}
\par \textbf{2 - } A distinção entre capital fixo e capital circulante em Marx não segue critérios físicos, mas,
econômicos. Ou seja, decorre do modo de circulação do valor (e não do valor-de-uso) das várias frações
do capital produtivo, KP.

\subparagraph{} \textbf{A)} Entre máquinas, equipamentos, instrumentos de trabalho, ferramentas, mão-de-obra,
matérias-primas, materiais auxiliares e produtos intermediários (semi-acabados), mostre, sob o
critério da circulação do valor, quais pertencem à categoria de capital circulante, e quais
pertencem á categoria de capital fixo;

\subparagraph{} \textbf{B)} Diferencie capital circulante e capital fixo tanto sob o ponto de vista qualitativo, quanto sob o
ponto de vista quantitativo.

\vspace{0.5cm}

\par \textbf{3 - } Suponha um capital no valor de C = c + v, adiantado no início do ano. Suponha também que este
1
capital reproduz-se a cada r meses por meio de um movimento completo D – M – D’, integralizando um
total de n rotações (ou seja, rodando n vezes) durante o período R de um ano (12 meses). O número de
rotações, n, que o capital efetua durante os 12 meses do ano, R, pode ser expresso de dois modos
diferentes: (a) n = R/r; ou (b) n = P/C, onde P é o produto anual acumulado excluindo a mais-valia.
Discuta as duas fórmulas, comparando-as. Invente exemplos para compreendê-las.

\vspace{0.5cm}

\begin{center}
\textbf{Sobre rotação do capital fixo}
\end{center}

\vspace{0.5cm}

\par \textbf{4 - } O tempo de rotação do capital fixo é determinado tecnologicamente, coincidindo com a duração
média do meio de trabalho em que se materializa. Porquê?

\vspace{0.5cm}

\par \textbf{5 - } Marx introduz um fator aparentemente contraditório com a determinação “tecnológica” do tempo de
rotação ou aquele dado pelo seu desgaste físico: é o seu “desgaste moral”. Discuta, comparando com o
seu conhecimento anterior, cada uma das questões abaixo: \footnote{Que denota a sua velocidade de rotação. Quanto menos tempo for r, maior é essa velocidade.}

\subparagraph{} \textbf{A)} Defina depreciação moral;

\subparagraph{} \textbf{B)} Porque a depreciação moral é uma forma de reprodução ampliada do capital?

\subparagraph{} \textbf{C)} Na representação contábil, o “fundo de amortização”, por um lado, e o “fundo de mais-valia”
ainda na forma de capital-dinheiro momentaneamente desocupado ou previamente destinado
ao consumo pessoal do capitalista, por outro, aparecem numa conta comum;



\subparagraph{} \textbf{D)} Se o desgaste moral do capital fixo é mais rápido que o seu desgaste físico, o empresário
comprime (reduz) o período de amortização, aumentando as cotas anuais que são imputadas
ao preço de custo, determinando a alta do preço do produto. Nisso consiste a “amortização”
acelerada, cuja prática falseia a objetividade do preço de custo calculado pela empresa, que deixa de representar apenas as condições de produção para refletir a estratégia de alocação do
excedente criado;

\subparagraph{} \textbf{E)} Na medida em que a cota de “amortização” passa a conter uma parcela da mais-valia, reduz-se
o lucro contábil;

\subparagraph{} \textbf{F)} O conceito de “fluxo de caixa” aparece para substituir o conceito de “lucro contábil” quando a
inclusão de uma parte da mais-valia na cota de “amortização” leva a certa confusão entre
excedentes e reservas para depreciação, reduzindo o “lucro contábil” na medida dessa
confusão. Assim, o “fluxo de caixa” torna-se, ao mesmo tempo, num indicador de rentabilidade
e de capacidade de acumulação da empresa.

\vspace{0.5cm}

\par \textbf{6 -} Marx nota que, na prática capitalista, a amortização não funciona de modo absolutamente conforme
à formação gradual de fundos de “amortização” relativo a cada tipo e unidade de capital fixo na medida
da entrada do respectivo valor em circulação, para, ao término de todo o período de amortização, obter
uma soma de valor que possibilite substituir o capital fixo velho por capital fixo novo. Discuta,
comparando com o seu conhecimento anterior, cada uma das questões abaixo:

\subparagraph{} \textbf{A)} Na prática, as cotas de amortização são utilizadas no período corrente, na renovação de outros
meios de trabalho ou seus componentes que encontram-se em final de vida útil. O valor da
depreciação de dada unidade de capital fixo não acumula-se gradualmente até tornar-se
possível substituí-lo ao atingir certo montante;

\subparagraph{} \textbf{B)} A especificidade da amortização contábil relativamente ao conceito de “fundo de reserva”
(inclusão de uma parcela da mais-valia, mobilização no período corrente, utilização na
reprodução ampliada) justifica a sua qualificação pela economia da empresa como “recurso de
autofinanciamento”, com estatuto idêntico à fração não-distribuída de lucro.

\vspace{0.5cm}

\par \textbf{7 -} O funcionamento do capital fixo no processo de reprodução imediato pressupõe a realização de
despesas de conservação e reparação, distintas da substituição (embora por vezes a fronteira entre
substituição e reparação seja difícil de traçar), cujo valor médio é o elemento constitutivo do capital
circulante e entra no valor do produto. Comente, com base nos seus conhecimentos.

\vspace{0.5cm}
\begin{center}
	\textbf{Sobre rotação do capital circulante}
\end{center}
\vspace{0.5cm}

\par \textbf{8 -} A rotação do capital circulante identifica-se com o movimento cíclico de reprodução do capital, sendo
equivalente à soma das suas fases de produção e de circulação. Comente.

\vspace{0.5cm}

\par \textbf{9 -} A estocagem dos meios de produção constitui-se no principal elemento de diferenciação dos tempos
de rotação das várias frações do capital circulante. Comente

\vspace{0.5cm}

\par \textbf{10 -} A frequência das compras de capital circulante depende, por um lado, da magnitude do estoque
existente, e por outro, do tempo que leva para esgotar-se ou ser completamente consumido na
produção. Comente.

\vspace{0.5cm}

\par \textbf{10 -} São determinantes do tempo de rotação do capital circulante: (a) o tempo de entrega dos
fornecimentos (associados a transportes e comunicações); (b) a dimensão relativa do estoque; e (c) a
velocidade de transformação de “capital produtivo latente” em meio de produção efetivo. Para dado
fluxo de transformação de capital circulante em produto-mercadoria, mostre:

\subparagraph{} \textbf{A)} Como um aumento em (a) afeta (b), supondo (c) constante?

\subparagraph{} \textbf{B)} Como uma redução em (b) afeta (a), supondo (c) constante?

\subparagraph{} \textbf{C)} Que implicações ramos de produção em que (c) é rápido em relação a ramos em que (c) é mais
lento trazem para (a) e (b)? 

\vspace{0.5cm}

\par \textbf{11 -} O tempo de rotação do capital circulante (que identifica-se com a duração do movimento cíclico do
capital) pode ser decomposto na duração das metamorfoses que integram o ciclo.

\subparagraph{} \textbf{A)} Defina e faça a distinção entre: (i) duração do processo de trabalho (período de trabalho),
considerando aqui as características dos diferentes ramos; e (ii) duração do processo de
produção (período de produção);

\subparagraph{} \textbf{B)} Faça a distinção entre: (i) duração do processo de produção (período de produção); e (ii)
duração do processo de circulação (tempo de circulação), considerando aqui as suas diferentes
metamorfoses – tempo de venda e tempo de compra;

\subparagraph{} \textbf{C)} Aponte os fatores que afetam o tempo de venda – período em que o capital se encontra em
estado KM;

\subparagraph{} \textbf{D)} Aponte os fatores que afetam o tempo de compra – período em que o capital se reconverte da
forma KD – KP;

\vspace{0.5cm}

\par \textbf{12 -} Explique como o tempo de rotação do capital circulante afeta as necessidades de adiantamento de
capital? (Use na resposta os conceitos de libertação e fixação de capital).

\vspace{0.5cm}
\begin{center}
	\textbf{Sobre mais-valia simples e massa anual de mais-valia}
\end{center}
\vspace{0.5cm}

\par \textbf{13 -} Sabendo que a mais-valia simples, m, consiste do montante de mais-valia obtido ao final de uma
rotação, defina a massa anual de mais-valia, M, relacionando-a com a mais-valia simples.

\vspace{0.5cm}

\par \textbf{14 -} Se a taxa simples de mais-valia é m’ = m/v, explique, qualitativa e quantitativamente, em que
consiste a taxa anual de mais-valia, M’ ou T, supondo que v é adiantado (sai do bolso do capitalista) na
primeira rotação, de uma vez para sempre, sem ser necessário proceder a novos adiantamentos
originários do bolso do capitalista.

\vspace{0.5cm}

\begin{center}
\textbf{Respostas:}
\end{center}


\begin{center}
\section{Rotação do Capital}
\end{center}


\par \textbf{1)} O conceito de rotação do capital se dá, pois não é um movimento isolado ou rígido, e sim frequente e em conjunto. Sendo bem semelhante ao estudo da circulação do capital. O objetivo do estudo da circulação é estudar a duração do ciclo global que este processo tem, para analisar o tempo que o capital gasta para essa circulação e as consequências para o capital produtivo, seus fatores e efeitos imediatos. 

\vspace{0.5cm}

\par \textbf{2)} 

\subparagraph{} \textbf{A)} Capital fixo: \textbf{Máquinas, equipamentos, instrumentos de trabalho e ferramentas.}
\\
Capital circulante: \textbf{Mão-de-obra, matérias-primas, materiais auxiliares e produtos intermediários.}

\subparagraph{} \textbf{B)} O capital fixo tende-se ao seu produto apenas um fração da mais-valia gerada pelo trabalho. O Capital circulante transfere a mercadoria, o seu valor total a elas e isso é demonstrado em cada processo de circulação.  As diferenças qualitativas podem se pensar da seguinte maneira, no capital circulante é renovado de forma continua, a partir das transformações KD (capital-dinheiro) e KP (capital-produtivo) pela compra.
\\
Já o capital fixo é substituído de forma descontinua, em períodos longos pelo desgaste dessas máquinas (depreciação), por isso, a amortização é altamente necessária no capital fixo. Já nas diferenças quantitativas, o capital fixo, o tempo de rotação desse capital é bem maior do que o circulante, tanto é que o Marx usa uma fórmula para definir esse processo.

\begin{equation}
n = \dfrac{P}{C}
\end{equation}
Sabendo que \textbf{n} é o número de rotações, \textbf{P} é o produto anual (mais-valia) e por fim \textbf{C} é o capital adiantado. Essa é a formula global que acontece as transferências de mais-valia.

\vspace{0.5cm}

\par \textbf{3)} A primeira fórmula:
\begin{equation}
n = \dfrac{R}{r}
\end{equation}
Essa é a expressão geral das rotações do capital, por ela, conseguimos definir a duração do ciclo, como também, divide as formas que o capital e volta como a funcionalidade. É definido que, \textbf{n} número de rotações anuais, como \textbf{R} unidade de tempo (ano), \textbf{r} como tempo de rotação. Um exemplo, se sabemos que R (12 meses) e também, sabemos que r é o tempo de rotação. Extrapolando que o nosso r é de 6 meses.

\begin{equation}
n = \dfrac{12}{6}
$$\\$$
n = 2
\end{equation}

Ou seja, o capital irá ter duas rotações anuais para ter continuação do processo.
\\
Agora, na nossa segunda fórmula: 
\begin{equation}
n = \dfrac{P}{C}
\end{equation}
Sabendo que P é o produto anual (retirando a mais-valia disso) e C, capital adiantado e por fim, n o número de rotações. Essa é a forma global.


\begin{equation}
n=\frac{n_{1} x_{1}+n_{2} x_{2}+\ldots+n_{h} x_{h}}{x_{1}+x_{2}+\ldots+x_{h}}=\frac{\sum_{i=1}^{h} x_{i} n_{i}}{\sum_{i=1}^{h} x_{i}}=\frac{P}{C}
\end{equation}

Já essa versão, é a média ponderada dos tempos de rotação dos vários elementos do capital produtivo. Demonstra uma divisão do capital circulante e fixo, variando por causa das rotações do capital, sabendo que \textbf{$n_{i}$} (Número de rotações do capital anual) e \textbf{$x_{i}$} (valor-capital), utilizando exemplos como acima, podemos definir nosso \textbf{P} em R\$ 120.000 e \textbf{C} em R\$ 60.000.

\begin{equation}
n = \dfrac{120.000}{60.000}
$$\\$$
n = 2
\end{equation}
Ou seja, a fatia que corresponde a mais-valia anual é de 2.

\begin{center}
\section{Rotação do Capital Fixo}
\end{center}

\par \textbf{4)} Isso acontece porque as varias frações do capital fixo, representadas por inúmeros trabalhos carregam diferentes tipos de formações e tempos e consequentemente, desgastes diferentes.

\vspace{0.5cm}

\par \textbf{5)} 

\subparagraph{} \textbf{A)} O desgaste moral é consequência da alta concorrência dos capitalistas para que os meios de produção sejam o mais eficiente possível, em busca de uma produtividade bem elevada.

\subparagraph{} \textbf{B)} É uma consequência da reprodução ampliada pois, em decorrência da constante ampliação de investimentos (d), isso gera uma nova ampliação, mostrando todo aquele processo de concorrência e produtividade. Essa reprodução ampliada gera um "fundo de reserva" e essa reprodução ampliada é feito pela mais-valia.

\subparagraph{} \textbf{C)} Nesse caso, podemos dizer que o desgaste moral é maior que o fisico, o capitalista irá capitalizar esse processo com aumentos anuais focadas no seu custo de fabricação, ocasionando uma alta do seu produto. 

\subparagraph{} \textbf{D)} Isso podemos argumentar que é um processo que acelera a amortização, distorcendo o cálculo do custo da empresa que já não corresponde as condições de produção (custos de produção) que irá apenas demonstrar o processo de acumulação da empresa e da apropriação do excedente criado.

\subparagraph{} \textbf{E)} Esse ponto é explicado no livro, porque esse processo acontece por causa do "cash-flow"(soma do lucro liquido e da amortização ) isso é uma indicação da rendibilidade e acumulação que a empresa possui. 

\subparagraph{} \textbf{F)} Para Marx, esse processo não funciona de um modo concreto, porque estão bem longe de haver uma acumulação gradual até realizar uma substituição do capital fixo, as quotas de amortização são utilizadas no período corrente para a atualização de outros processos de trabalho. Ou seja, esse fundo de amortização completa o valor do capital fixo, usando uma quantidade que foi gerado por um capital fixo, porém, agora é gerado uma quantidade maior ou até mesmo, mais eficaz.

\vspace{0.5cm}

\par \textbf{6)} 

\subparagraph{} \textbf{A)} Esse processo é chamado de "fundo de reserva", essas quotas de amortização são utilizadas no período corrente para a renovação de outros meios de trabalho. Ou seja, esse "recurso de autofinanciamento" é usado para a renovação, ou seja, são usados inúmeras frações desses processos de amortizações são usadas para a renovar o capital.

\subparagraph{} \textbf{B)} Esse funcionamento do capital fixo requer que o processo de produção pressupõe a realização de despesas de conservação e reparação, distintas da substituição, o valor médio é um elemento que compõe o capital circulante e entra no valor do produto.

\vspace{0.5cm}

\par \textbf{7)} Um ponto que já debatemos nas questões anteriores é a questão que a construção do capital fixo é oriundo do capital circulante, como todo o processo de amortização e etc. OU seja, para a construção do capital fixo usamos pequenas frações que são realizadas do processo de circulação e o capitalista retira pequenos pedaços para a reinstituição do capital fixo.

\vspace{0.5cm}


\begin{center}
\section{Rotação do Capital Circulante}
\end{center}

\par \textbf{8)} Essa é uma afirmação interessante, pois o capital circulante leva em conta dois processos importantes para Marx, que é o da produção e circulação. Porém, essa rotação é não uniforme, pois o tempo dessa rotação é bem diferente no processo, processos de produção e de circulação tem suas particularidades e diferenças, como citado no texto usando os exemplos do algodão, carvão e ferro, são processos distintos. Ou seja, o capitalista tem que saber lidar com estoques e seus encargos, como também ter a noção de uma utilização ideal e pensar na sua produção num ponto ótimo, afinal, ter um estoque em grandes quantidades correspondem a prejuízos ao próprio.

\vspace{0.5cm}

\par \textbf{9)} Usando o exemplo do texto, algodão e carvão, o tempo de uso e a utilidade dessas matérias tem um período de tempo bem diferente do padrão, logo, essas matérias tendem a ser substituídas de forma frequente, não necessitam de uma compra quase diária, a compra delas são definidas a partir do estoque que o capitalista possui e o tempo que esses produtos podem ser esgotarem, bem diferente da força de trabalho que não tem um estoque dessa forma.

\vspace{0.5cm}

\par \textbf{10)} Um dos motivos para esse processo é o tempo que essas matérias são entregues, o tamanho do estoque e a velocidade que essas matérias tendem a se transformar em capital-produtivo latente são diferentes e prazos também não tem uma sincronia. A junção desse processo é de uma compreensão mais complexa e pesada, pois, o tempo de rotação do capital variável torna-se bem desconexa e no interior dessas rotações tornam o processo com um delay na produção e a sua finalidade (causado pela diferença  das matérias primas e auxiliares).

\vspace{0.5cm}

\par \textbf{10)} 

\vspace{0.5cm}
\subparagraph{} \textbf{A)} Se houver um aumento de \textbf{a}, haverá uma demora para entrega dessas matérias-primas, logo, \textbf{b} tem que estoque suficiente para manter o nível de produção já que sabemos que não a alteração de \textbf{c}. Uma conclusão desse processo é que se houver um atraso crucial e chegar a ter uma falta nos estoques, isso irá afetar a velocidade da transformação do capital.

\subparagraph{} \textbf{B)} A redução de \textbf{b} gera uma pressão para ampliação de \textbf{a}, forçando uma ampliação de novas matérias primas para manter a produção do capital latente, o efeito será uma demanda maior para entrega de matérias primas e em tempo menor.

\subparagraph{} \textbf{C)} Se o tempo de \textbf{c} for rápido irá gerar um choque de demanda em \textbf{a} para entregas e um tempo menor e o efeito no \textbf{b} será crucial também, já que os estoques terão de ser restabelecidos de forma inteligente, para não gerar ônus de uma grande estocagem, porém, como o \textbf{c} estará com uma etapa de produção rápida deverá fazer que as entregas sejam frequentes e rápidas e os estoques terem um restabelecimento dinâmico.
\\
Agora, se \textbf{c} for lento, irá gerar uma entrega de \textbf{a} menor e com um tempo maior, afinal, os estoques \textbf{b} não poderão ter um estoque tão grande já que é um prejuízo estoques enormes, logo, com um \textbf{c} lento, as demais \textbf{a} e \textbf{b} terão uma rotação menor. 

\vspace{0.5cm}

\par \textbf{11)} 


\subparagraph{} \textbf{A)} A duração do processo de trabalho (período de trabalho), possuem dois setores: Setor \textbf{a} possuem trabalho longo e continuo e um produto indivisível (ex:material ferroviário), em que a venda do produto se processa em intervalos longos. Setor \textbf{b} é o setor de tempo curto e descontinuo e de produto divisível, em que a transformação de KM e KD se processa constantemente.
\\
Agora, o período de produção, é um processo  que não contém apenas o processo de trabalho, mas também pelas suas interrupções, como é um processo que tem a característica da coexistência dos trabalhos aqui com os limites que este mesmo impõe.

\subparagraph{} \textbf{B)} O processo de circulação (tempo de circulação) é um processo que contempla dois processos claros, capital mercadoria e capital dinheiro, cujo as determinações são feita pelo "tempo de venda" e "tempo de compra".
\\
O tempo de venda é caracterizado por alguns fatores como flutuações sofridas pela situação de mercado, para conjunto da economia ou setor especifico. Pelo processo de transporte, entre o local de produção e o mercado, ou seja, a geografia é crucial para a definição de transportes, ou seja, o tempo de venda nesse quesito tem uma correlação com as vias de transportes e a localização industrial. O volume de crédito aos clientes (prazo de pagamentos) e por fim, em indústrias de tempo curto o tempo de venda varia conforme a quantidade que é vendida, sejam de uma vez ou por quantidades definidas e isso define a forma que a empresa irá funcionar.
\\
O tempo de compra é o movimento que o capital-dinheiro se transforma no capital-produtivo, é definido pelo tamanho e escoamento dos estoques - ou seja, o capital produtivo latente que define este processo. Temos duas variáveis nesse tempo de compra, a dimensão do estoque é variável, depende muito da matéria prima e a sua disponibilidade de mercado e outra variável, é a rotação do estoque que não demonstra apenas a latência da produção como também o ritmo da realização do produto (tempo de venda).

\subparagraph{} \textbf{C)} Os fatores que afetam são as flutuações econômicas, sejam pela economia ao todo ou por alguns setores. Problemas com fretes e local da empresa, como montar uma rede de logística para que o tempo de venda funciona de forma correta. O crédito que os clientes tem para essas compras e os prazos de pagamentos e por fim, a forma que as empresas vendem esses produtos - sejam de uma vez ou por quantidades pré-definidas. 

\subparagraph{} \textbf{D)} Os dois processos que afetam o tempo de compra são o tamanho do estoque, conforme a matéria prima e a disponibilidade que o mercado produz e por fim, a rotação do estoque que define não só a latência do capital produtivo e também o ritmo de tempo de compra do produto.

\vspace{0.5cm}

\par \textbf{12)} Esse processo pode ter alguns problemas conforme é transporto no livro, como por exemplo. O tamanho do capital adiantado da produção é calculado a partir da rotação do capital circulante, enquanto o capital fixo é um processo longo e demorado, o circulante é continuo (salários e matérias-primas). Outro processo que define como o tempo de rotação do capital circulante é afetado, no capital produtivo não possuem a disponibilidade de obter crédito, resta ao capitalista usar o capital adiantado para pagar o capital circulante para manter esse ciclo.
\\
Assim, temos a "liberação de capital" e "fixação do capital", demonstra que uma indústria que produz mercadorias que tem um tempo de produção maior necessita de um capital adiantado maior que a indústria de produtos com tempo menores, então, se pensarmos que o tempo de venda se modifica a partir das conjunturas do mercado, se o tempo de circulação diminui ocorre a "libertação do capital" que é usado para garantir a continuidade dos ciclos e precisamente o ciclo produtivo, agora se o processo for diferente ocorre a "fixação do capital".

\vspace{0.5cm}

\begin{center}
\section{Mais-valia simples e Massa anual de mais-valia}	
\end{center}


\par \textbf{13)} A massa anual de mais-valia corresponde ao excendente gerado por \textbf{m}, é realizado por toda a força de trabalho empregada numa rotação, é uma acumulação de todas as rotações \textbf{n} anuais.

\begin{equation}
M = n \cdot m
$$\\$$
\text{ou}
$$\\$$
M = m'Y
\end{equation} 

A massa de taxa mais-valia simples é composta de \textbf{n} unidades de força de trabalho realizados simultaneamente. A diferença é que massa de mais-valia anual é composta por realizações anuais por causa das rotações que o capital realiza.

\vspace{0.5cm}

\par \textbf{14)} Lembrando que a taxa de mais-valia é m' = $\dfrac{m}{v}$, ou seja, ela mede o valor do excedente m'(tempo de trabalho excedente) sobre custo de produção (tempo de trabalho necessário), assim definimos a taxa de mais-valia simples. A taxa anual de mais-valia é a medida do excedente contando com as rotações que o capital realiza ou seja, m' = $\dfrac{m}{v}$ porém agora estamos trabalhando com um processo mais amplo e definido pelas rotações que ocorre num ano. De forma qualitativa, sabemos que a mais-valia é que ele tem a sua determinação pelo capital circulante e a qualidade de mão de obra para gerar esse valor, como também a produtividade que o trabalhador irá empregar nesse setor e de forma quantitativa é o volume que essas rotações irá empregar na mais-valia e na sua taxa anual, dependendo de como funcionar o seu m e v, e se o capitalista conseguiu gerar um bom funcionamento das suas capacidades produtivas. 
\\
Um exemplo numérico, se o capitalista tiver o investimento em capital de 4.000, e sabendo que ele usará 4/5 em capital constante no valor de 3.200 e o restante 1/5 800 no capital variável, podemos pensar da seguinte maneira, ele gasta 4 semanas para a produção, 3 na produção e 1 na circulação. Então, sabemos que ele gasta 38.400 para adiantar o seu capital variável (48 s x 800) num ano todo.
\\
Então, podemos tentar medir as rotações que esse capital faz $n = \dfrac{38.400}{3.200}$, ou seja, esse capital dá 12 rotações, então, é possível afirmar que o capital anual é da seguinte maneira 4/5, 30.720 para o capital constante e 7.680 para o variável, o correspondente ao 1/5. Se cogitarmos uma taxa de mais-valia de 100\% esse capital variável nos gerará outros 800 de mais-valia, fazendo alguns cálculos com os números de rotações que temos (12), saberemos que esses 9.600 gerados em mais-valia ao capitalista. Ou seja, o adiantamento do capitalista é crucial para contratação de força de trabalho, porém, dependendo da sua taxa de mais-valia ele consegue captar ainda mais excedente dos seus funcionários, utilizando os cálculos anuais de rotações conseguimos buscar uma compreensão desse ciclo.

\end{document}
