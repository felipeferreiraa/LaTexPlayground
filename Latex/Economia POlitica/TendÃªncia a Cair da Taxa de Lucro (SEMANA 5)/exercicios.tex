\documentclass[a4paper, 12pt]{article} %Tipo de documento.
\usepackage[top=2cm, bottom=2cm, left=2.5cm, right=2.5cm]{geometry} %Pacote para Margens
\usepackage[utf8]{inputenc} % Pacote para permitir a utilização de acentos.
\usepackage{amsmath, amsfonts, amssymb, dsfont} % % Pacote matemático
\usepackage{graphicx} % Pacote inserir imagens.
\usepackage[portuguese]{babel} % Pacote para usar diversos idiomas
\usepackage{float} %Para inserir textos nas imagens
\usepackage{tikz} % Pacote para gráficos Tikz
\usepackage{tabu} %Pacote para tabular
\usetikzlibrary{arrows} % Pacote para o Tikz
\usepackage{setspace} %Pacote para espaçamento entre linhas
\usepackage{cases} % Para definir ordens ao modelo matemático
\usepackage[normalem]{ulem} % Pacote para usar o uuline (texto sublinhado)
\usepackage{multirow} % Pacote para criação de Tabular.
\usepackage{tikz-3dplot}

\begin{document}

\begin{center}
\textbf{UNIVERSIDADE FEDERAL DO TOCANTINS\\
	CAMPUS UNIVERSITÁRIO DE PALMAS\\
	COORDENAÇÃO DE ECONOMIA}
\end{center}

\textbf{Prof: Fernando Jorge Fonseca Neves }
\singlespacing
\textbf{Aluno: Felipe Ferreira de Sousa}
\begin{center}
\textbf{Atividade 5: Tendência a Cair da Taxa de Lucro}
\end{center}
\vspace{0.5cm}

\par \textbf{1 - } Foi visto que - a despeito das diferentes composições orgânicas do capital não só entre os
diferentes capitais individuais no interior de cada ramo mas também entre os diferentes ramos
de produção – a concorrência intercapitalista produz uma taxa de lucro uniforme para todos
os ramos ou taxa geral de lucro (ou lucro médio). Foi visto, também, que este mesmo processo
que faz originar a taxa geral de lucro (média) converte valores de produção em preços de
produção.
Por outro lado, uma vez formada tal taxa de lucro, ela tende a cair à medida em que as forças
produtivas capitalistas desenvolvem-se – fenômeno que foi traduzido na forma de “Lei da
tendência a cair da taxa de lucro”. Explique os fundamentos teóricos desta “Lei”.

\vspace{0.5cm}

\par \textbf{2 -} Apesar da tendência subjacente à “Lei”, os mesmos princípios que a fundamentam
produzem, também, fatores de “contra-tendência”, fazendo com que a “Lei” não se realize,
podendo, inclusive, inverter a tendência. Aponte cada um destes fatores e explique como
operam na inversão da tendência. Discuta sobre a corrente marxiana que aponta esta “Lei”
como fator de crise do Modo de Produção Capitalista

\vspace{0.5cm}

\begin{center}
\textbf{Respostas:}
\end{center}

\section{Tendência a Cair da Taxa de Lucro}

\par \textbf{1 -} Baseado nos textos anteriores já definimos os custos para os capitalistas, como os seus lucros médios baseado numa massa total de lucro que todos os setores da economia produzem, logo, essa tendência gera alguns fatores para a análise marxista. Um fator que isso irá gerar é a competição acelerada em busca de lucros exponenciais e o máximo que cada capitalista consegue tirar desta competição intracapitalista. Essa competição é um dos maiores catalisadores da produtividade, lembrando que o progresso técnologico está atrelado ao menor tempo possível para gerar lucros, isso vai implicar o investimento massivo em capital constante (máquinas, indústrias, etc). Para o Gonzalez esse é um dos maiores fatores para afastar os operários da c.o.c, pois, já não vai haver investimentos tão grandes no capital variável, portanto vai haver um aumento um na nossa fórmula $\dfrac{c}{v}$ em termos constantes, pois o trabalho vivo irá necessitar de alguns meios de produção para o seu funcionamento.
\\
A tendencia de uma taxa de lucro menor é um efeito claro da produtividade que ocorre na produção, consequência da competitividade capitalista, ou seja o acumulo da mais-valia extraordinária é um causador do decréscimo da taxa de lucro, já explicada na c.o.c. Podemos afirmar alguns fatores para a queda da taxa de lucro, é que a acumulação dos lucros anteriores fazem que ocorram um aumento no capital social de forma escalar. Ou seja, temos a noção de que se a massa absoluta de lucro aumente o volume total do capital deve crescer a ritmos rápidos que os da diminuição da taxa de lucro.

\vspace{0.5cm}

\par \textbf{2 -} Dentre todos os fatores que contrariam estas leis, a que mais nos é explicita é o fator é o aumento da exploração do trabalhador, aumento da taxa de mais-valia condicionada ao aumento da produtividade do trabalhador. O aumento da taxa de mais-valia condiciona algumas coisas como na c.o.c irá ter um uma quantidade maior da massa de mais-valia, como também haverá um decréscimo da taxa de lucro, determinada pela c.o.c, ou seja, se m' aumentar mais que o dobro, a margem de lucro subirá em vez de descer, apesar do aumento na composição orgânica do capital.
\\
Outro fator que interfere nas leis é uma da qual o Marx descreve bem, é que todas as mercadorias, incluindo a força de trabalho são vendidas pelo seu valor. Isso inclui também a tendência dos capitalistas em pagarem salários abaixo da força de trabalho, gerando um exército de reserva para que os salários estejam abaixo do seu suposto ponto ótimo, o que obriga os salários abaixo do ponto ótimo. Outro fator crucial é o barateamento dos meios de produção ocasionada pela produtividade dos setores, devido esse barateamento, a composição técnica do capital desenvolve-se mais rapidamente que a sua composição de valor, o que atenua os ritmos do crescimento da composição orgânica e que por consequência diminuem os lucros.
\\
O comércio externo também é um fator que contribui para a tendência á baixa da taxa de lucro, quando, por seu intermédio se torna mais barato ter meios de produção barateados. Por fim, podemos definir alguns detalhes como, a lei tendencial de baixa de lucros  não se manifesta na diminuição, mas, no seu crescimento em menor escala da taxa de mais-valia, o que demonstra como funciona o desenvolvimento do capitalismo.

\end{document}