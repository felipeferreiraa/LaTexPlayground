\documentclass[a4paper, 12pt]{article} %Tipo de documento.
\usepackage[top=2cm, bottom=2cm, left=2.5cm, right=2.5cm]{geometry} %Pacote para Margens
\usepackage[utf8]{inputenc} % Pacote para permitir a utilização de acentos.
\usepackage{amsmath, amsfonts, amssymb, dsfont} % % Pacote matemático
\usepackage{graphicx} % Pacote inserir imagens.
\usepackage[portuguese]{babel} % Pacote para usar diversos idiomas
\usepackage{float} %Para inserir textos nas imagens
\usepackage{tikz} % Pacote para gráficos Tikz
\usepackage{tabu} %Pacote para tabular
\usetikzlibrary{arrows} % Pacote para o Tikz
\usepackage{setspace} %Pacote para espaçamento entre linhas
\usepackage{cases} % Para definir ordens ao modelo matemático
\usepackage[normalem]{ulem} % Pacote para usar o uuline (texto sublinhado)
\usepackage{multirow} % Pacote para criação de Tabular.
\usepackage{tikz-3dplot}

\begin{document}

\begin{center}
\textbf{UNIVERSIDADE FEDERAL DO TOCANTINS\\
	CAMPUS UNIVERSITÁRIO DE PALMAS\\
	COORDENAÇÃO DE ECONOMIA}
\end{center}

\textbf{Prof: Fernando Jorge Fonseca Neves }
\singlespacing
\textbf{Aluno: Felipe Ferreira de Sousa}
\begin{center}
\textbf{Atividade 4: Relação entre Taxa de Lucro, Composição do Capital Adiantado e Taxa de Mais-valia}
\end{center}
\vspace{0.5cm}

\vspace{0.5cm}

\par \textbf{1 -} A composição orgânica do capital ou do investimento C = c + v constitui-se numa medida da importância de cada uma das duas formas do capital – capital constante (c) e capital variável (v) – no capital total adiantado C. Sendo c e v duas partes disjuntas na composição de C, então, para dado C, aumentos em c representam reduções em v e vice-versa. Por outro lado, se C não estiver fixado, mas, v estiver, então, acréscimos em c levam a acréscimos iguais em C, resultando num aumento na importância de c na composição do capital C. Invertendo, se c estiver fixado, mas, v não estiver, acréscimos em v levam a acréscimos iguais em C, resultando num aumento na importância de v na composição do capital C. 

\subparagraph{} \textbf{A)}  Define-se progresso tecnológico como o processo pelo qual se amplia a base de
produtividade, ou seja, a mesma quantidade de trabalho passa a dar origem a quantidade
crescente de valores de uso ou unidades de produto. Em termos da composição orgânica
do capital, o progresso tecnológico se expressa em acréscimos na importância do capital
constante, c, na composição do capital total, C, o que também representa ganhos de
importância do capital constante, c, relativamente ao capital variável, v; ou, de modo
invertido, se expressa em perda de importância do capital variável, v, no capital total, C, o
que também representa perda de importância do capital variável, v, relativamente ao
capital constante, c. A partir da ideia de ganhos de produtividade, explique estes efeitos do
progresso tecnológico na composição orgânica do capital.

\vspace{0.5cm}

\par \textbf{2 -}  A taxa de mais-valia, m’ = m/v, é, para Marx, uma medida do grau de exploração do trabalhador. Ao
definir a mais-valia, m, como o montante de excedente (dado em magnitude de valor) que uma unidade
de força de trabalho (Ft, de custo v) pode produzir como capital, Marx encontra a necessidade de definir
a medida relativa desse excedente com base no custo de obtenção do seu elemento gerador, v, isto é,
de saber o montante de mais-valia produzida por unidade de gasto, v. Ao mesmo tempo, a taxa de mais-valia constitui-se numa medida de distribuição de rendimentos ou do valor novo produzido, v + m.
Explique.

\vspace{0.5cm}

\par \textbf{3 - } A taxa de lucro pode ser expressa como $I = \dfrac{m}{c + v}$ $\Rightarrow$ $I' = \dfrac{m}{\frac{c}{v} + 1}$ $\Rightarrow$ $I' =m \cdot \dfrac{v}{c + v}$  Considere cada
uma das situações relativas ao trabalho, apontadas nas linhas abaixo, e discuta que impactos impõem à taxa de lucro:

\subparagraph{} \textbf{A)} Supondo constante o valor da força de trabalho (ou o valor do salário), a quantidade de
trabalho contratada, e o valor dos meios de produção; e ocorre um aumento (ou redução) na
duração da jornada de trabalho;

\subparagraph{} \textbf{B)} Supondo constante o valor da força de trabalho (ou o valor do salário), a duração da jornada de
trabalho, e o valor dos meios de produção; e a quantidade de trabalho contratado (número de
trabalhadores) aumenta (ou diminui);

\subparagraph{} \textbf{C)} Supondo constante a duração da jornada de trabalho, a quantidade de trabalho contratada, e o
valor dos meios de produção; e o valor unitário da força de trabalho, v, aumenta (ou diminui);

\vspace{0.5cm}

\par \textbf{4 -} Considere cada uma das situações relativas à composição orgânica do capital, $c.o.c = \dfrac{c}{v}$ ou,
inversamente, $c.o.c = \dfrac{v}{C}$, apontadas nas alíneas abaixo, supondo constante a taxa de mais-valia, m’.
Discuta que impactos impõem á taxa de lucro e demonstre:

\subparagraph{} \textbf{A)} Supondo o capital constante, c, fixo, ao longo de dois momentos consecutivos, um aumento na
proporção do capital variável, v, relativamente ao capital total, C = c + v, por meio de dois
casos: (i) o capital total, C, não se modifica; e (ii) o capital total, C, modifica-se no montante de
variação do capital variável, v;

\subparagraph{} \textbf{B)} Supondo o capital variável, v, fixo, ao longo de dois momentos consecutivos, um aumento na
proporção do capital constante, c, relativamente ao capital total, C = c + v, por meio de dois
casos: (i) o capital total, C, não se modifica; e (ii) o capital total, C, modifica-se no montante de
variação do capital constante, c;

\vspace{0.5cm}

\par \textbf{5 -}  Supondo agora constante a composição orgânica do capital, c.o.c., discuta os efeitos de aumentos /
reduções na taxa de mais-valia, m’, sobre a taxa de lucro, lembrando de considerar que a taxa de maisvalia é uma medida de distribuição de renda entre capital e trabalho.

\vspace{0.5cm}

\par \textbf{6 -} Supondo, agora, que tanto a taxa de mais-valia, m’, quanto a composição orgânica do capital, c.o.c.,
são modificáveis. Como discutir a taxa de lucro neste contexto?

\vspace{0.5cm}

\par \textbf{7 - }  A ocorrência de acréscimos na proporção de capital constante relativamente ao capital total, próprio
do progresso tecnológico, faz reduzir a taxa de lucro. Mas, se tal progresso tecnológico for aplicado aos
ramos que produzem os meios de consumo dos trabalhadores, reduz-se o valor desses meios e,
consequentemente, o valor da própria força de trabalho - aumentando, assim, a taxa de mais-valia.
Discuta este complexo efeito sobre a taxa de lucro.

\vspace{0.5cm}

\begin{center}
\textbf{Respostas:}
\end{center}

\section{Relação entre Taxa de Lucro, Composição do Capital Adiantado e Taxa de Mais-valia}

\par \textbf{1 -}

\subparagraph{} \textbf{A)} O progresso tecnológico está intrinsecamente atrelado ao capital constante e variável, vejam que as indústrias carecem de uma constante mão-de-obra ou máquinas habilitadas para manterem a sua produção em níveis adequados e essenciais para criar produtos e derivados em tempo mínimo.
\\
Essa produtividade tem por finalidade gerar lucros mais rápidos ao capitalista, o investimento em capital constante é necessário, já que máquinas e edifícios tem a sua importância máxima para geração das mercadorias, é um ramo de investimento que tem dois fatores intrínsecos como - valor e a velocidade de geração de valor. Primeiro, o valor é característico do capital constante, pois, máquinas e edifícios são caríssimos. Porem, da composição orgânica esse é o setor que mais acrescenta a produtividade. 
\\
No capital variável, temos a ideia das matérias primas e dos trabalhadores, mas, vejamos com cautela, pois essa parte da composição nos gera o ganho da mais-valia - principalmente, por causa dos trabalhadores. Os trabalhadores são essenciais para o funcionamento do sistema e também, geram produtividade ao capitalista na sua saga a obtenção de lucros, vejamos da seguinte maneira, em tempos mais iniciais da revolução industrial, sabíamos que os trabalhadores entregavam a sua mão-de-obra por horas longas, casos de operários que ficavam 12 a 16 horas trabalhando em máquinas para gerar um efeito de produtividade máximo ao capitalista.
\\
Ou seja, a produtividade oriunda dessa alta quantidade de trabalho não é benéfica ao longo-prazo, imaginamos o desgaste que foram gerados aos trabalhadores. Logo, a evolução dos maquinários foram cruciais para que o capital variável fosse usado de maneira mais consciente e equilibrado, os capitalistas costumam gastar uma quantia bem superior no capital constante do que o variável, e isso gera um progresso tecnológico na produção de maneira crucial, diminuindo o desgaste do capital variável e o tempo de trabalho, mas, isso não significa perda de lucros e sim, ganhos de tempo, o que o empresário busca e um tempo de geração de lucro menor. Por fim, conseguimos determinar que o importante para a composição do capital orgânico tem ganhos exponenciais com a evolução do capital constante.

\vspace{0.5cm}

\par \textbf{2 -} A taxa de mais-valia já definida por Marx, tem a sua fórmula dada por $m' = \dfrac{m}{v}$, e por ela podemos definir por exemplo o grau de exploração que o trabalhador realiza para gerar a mais-valia ao capitalista. É de se lembrar que essa taxa de mais-valia está ligada ao KP (Capital produtivo), logo, para definir a criação do valor novo produzido devemos ter a noção de que essa relação de $m' = \dfrac{m}{v}$ é primordial para a criação do valor, por exemplo se nessa relação, tivermos a noção de que a taxa de mais-valia é 100\%, logo o nosso valor novo será destes 100\%, um exemplo levando que o a nossa taxa será essa, m = c + v + m, assim: m = 750 + 320 + 320, ou seja, a taxa irá nos retornar o excedente que é gerado no capital variável (v). Por isso, definimos que essa taxa de mais-valia também é de distribuição de valor novo, afinal, ela define o tamanho do nosso excedente.

\vspace{0.5cm}

\par \textbf{3 -}

\subparagraph{} \textbf{A)} A situação será a de $I' = m \cdot \dfrac{v}{c + v}$, assim, para ocorrer alterações nessa fórmula e principalmente, com a jornada de trabalho, essa será a formula.

\subparagraph{} \textbf{B)} A situação será a seguinte: $I' = \dfrac{m}{\frac{c}{v} + 1}$, assim, vemos que na fórmula, o que vai determinar a ocorrência desse fenômeno é o $\dfrac{c}{v}$ do denominador.

\subparagraph{} \textbf{C)} 

\vspace{0.5cm}

\par \textbf{4)} 

\subparagraph{} \textbf{A)} Nesse caso, a composição orgânica do capital será da seguinte maneira: $c.o.c = \dfrac{c}{v}$, se o capital fixo, for constante isso irá incidir em movimentos compensatórios de c, em sentido contrário de v, o que implica em maiores movimentos de v. Ou seja, o diferencial entre a taxa de lucro e a taxa de mais-valia é igual a proporção do capital variável no capital total, os dois casos serão:

\begin{equation}
I'_{1} = m'_{1}\dfrac{v_{1}}{c}
$$\\$$
I'_{2} = m'_{2}\dfrac{v_{2}}{c}
$$\\$$
\dfrac{I'_{2}}{I'_{1}} = \dfrac{m'_{2}\frac{v_{2}}{c}}{m'_{1}\frac{v_{1}}{c}}
$$\\$$
\dfrac{I'_{2}}{I'_{1}} = \dfrac{m'_{2}\cdot v_{2}}{m'_{1} \cdot v_{1}}
$$\\$$
m = m' \cdot v, \text{obtém-se} \dfrac{I_{2}}{I_{1}} = \dfrac{M_{2}}{M_{1}}
\end{equation} 

Podemos pensar em algumas particularidades dessas demonstrações, por exemplo, \textbf{v e m' variam em sentidos opostos}, se  ocorrer um aumento de v, sem aumentar as horas de trabalho, o montante de valor novo não irá se alterar, apenas haverá uma alteração proporção do valor novo é distribuído entre capital (m) e trabalho (v). Logo, m' varia em sentido oposto a v, então se C = c + v for constante, portanto, c deve ser reduzido para se adequar a v.
\\
E por fim, \textbf{m' e v movimentarem no mesmo sentido} se o aumento de v decorrer de ganhos salarias e também horas de trabalho, m' poderá gerar um excedente maior que corresponde aos acréscimos do salário, nesse caso, haverá ganhos reais.

\subparagraph{} \textbf{B)} Neste caso, teremos o seguinte parâmetro:

\begin{equation}
1) \ I'_{1} = m'_{1}\dfrac{v}{c}
$$\\$$
I'_{2} = m'_{2}\dfrac{v}{c}
$$\\$$
2) \ \dfrac{I'_{2}}{I'_{1}} = \dfrac{m'_{2} \frac{v}{c}}{m'_{1}\frac{v}{c}}
$$\\$$
\dfrac{I'_{2}}{I'_{1}} = \dfrac{m'_{2}}{m'_{1}}
\end{equation}
Ou seja, esses dois casos refletem quando ocorre uma fixação do capital variável, o que ocorre é que quando c + v for constante, isso gerará em um reajuste do capital investido para gerar novos ganhos, já que o variável está fixado. Outro ponto, é que este capital fixo pode ter um novo ponto de progresso técnologico é que irá gerar ganhos de produtividade e com o capital variável fixo, teremos um lucro maior já que não haverá tanto gastos com salários.

\vspace{0.5cm}

\par \textbf{5 -} Se a composição orgânica do capital for constante $\dfrac{c}{v}$ ocorrerá de que a proporção real do seu excedente ser determinada não pela relação do capital variavel, mas sim com a relação da taxa de lucro, excluindo a taxa de mais-valia. A taxa de lucro, é portanto produto da relação entre o excedente m com o capital adiantado, c + v:

\begin{equation}
I' = \dfrac{m}{c + v} 
$$\\$$
\text{ou}
$$\\$$
I' = \dfrac{m}{v}
\end{equation}
Portanto, neste caso, é de se afirmar que vai haver um problema sério, pois, o excedente será oriundo do capital e não do trabalho que é investido. Neste caso, o trabalho morto será o fator que irá criar o excedente do capitalista, isso geraria numa taxa de lucro enorme, porém, para Marx esse é um caso que não existe.

\vspace{0.5cm}

\par \textbf{6 -} Esse é o caso que a composição orgânica do capital corresponde a realidade para Marx, afinal, as modificações do c.o.c significa que o capitalista irá determinar como trabalhar de maneira realista com capital variável e constante, o capitalista conseguirá determinar como irá obter o seu excedente, se ele quiser investir mais em c, em busca de uma produtividade e diminuir o v, haverá uma flexibilidade maior deste empresário para definir, por exemplo qual será a taxa de exploração do trabalhador.

\vspace{0.5cm}

\par \textbf{7 -} Portanto, depende da variação na composição entre essas diferentes partes do capital adiantado, o aumento do capital constante apenas ocasiona em crescimento na magnitude do capital total, sem afetar a quantidade de valor excedente produzido, o que gera uma queda na taxa de lucro. Já o aumento do capital variável, ocasiona um aumento do capital adiantado mas, também irá gerar um aumento no excedente do capitalista.

\end{document}